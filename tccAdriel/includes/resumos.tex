% ---
% RESUMOS
% ---

% resumo em português
\setlength{\absparsep}{18pt}
\begin{resumo}
    A modernização das práticas agrícolas tem se mostrado essencial para aprimorar a precisão e a eficiência na coleta de dados em diferentes culturas como, por exemplo, o cacau. Este trabalho explora a integração da tecnologia de Identificação por Radiofrequência (RFID) ao aplicativo ColetaCacau, desenvolvido pela Universidade Estadual de Santa Cruz (UESC) em parceria com a Comissão Executiva do Plano da Lavoura Cacaueira (CEPLAC). A aplicação do RFID ao ColetaCacau visa automatizar a identificação das árvores, minimizando significativamente o erro humano e otimizando o tempo e os recursos empregados no campo. Essa tecnologia oferece uma solução robusta e econômica ao permitir que cada árvore monitorada seja identificada de forma rápida e precisa por meio de etiquetas RFID, mesmo em áreas remotas com pouca ou nenhuma conectividade. Além disso, o ColetaCacau com RFID elimina a necessidade de identificação manual e o uso de planilhas físicas, proporcionando um processo de coleta contínuo e altamente confiável que impacta diretamente a produtividade dos coletores.
    \vspace{\onelineskip}
    
    \noindent 
    \textbf{Palavras-chave}: rfid, coleta de dados, agricultura de precisão, ColetaCacau, tecnologia móvel, aplicativo híbrido, react native, sincronização offline, realmdb, cacau.
\end{resumo}

% resumo em inglês
\begin{resumo}[Abstract]
 \begin{otherlanguage*}{english}
     The modernization of agricultural practices has proven essential to enhancing precision and efficiency in data collection for high-value crops such as cocoa. This work explores the integration of Radio Frequency Identification (RFID) technology into the ColetaCacau application, developed by the State University of Santa Cruz (UESC) in partnership with the Executive Commission of the Cocoa Crop Plan (CEPLAC). The application of RFID to ColetaCacau aims to automate tree identification, significantly reducing human error and optimizing time and resources in the field. This technology offers a robust and cost-effective solution by allowing each monitored tree to be identified quickly and accurately via RFID tags, even in remote areas with little or no connectivity. Furthermore, ColetaCacau with RFID eliminates the need for manual identification and physical spreadsheets, providing a continuous and highly reliable data collection process that directly impacts the productivity of field workers.
    \vspace{\onelineskip}
    
    \noindent 
    \textbf{Keywords}: rfid, data collection, precision agriculture, ColetaCacau, mobile technology, hybrid application, react native, offline synchronization, realmdb, cocoa.
 \end{otherlanguage*}
\end{resumo}
