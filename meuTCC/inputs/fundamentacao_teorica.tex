\chapter{Fundamentação Teórica}

\section{Funcionamento do coração}
\label{sec:funciionamento_coracao}

O coração é um órgão muscular composto por quatro câmaras — átrio direito e esquerdo, e ventrículo direito e esquerdo — que se contraem de forma rítmica, bombeando sangue para o corpo. Essas contrações são controladas por correntes elétricas que percorrem o coração de maneira precisa e em velocidade controlada.

Na figura \ref{fig:coracao_esquema_eletrico}, o sistema de condução elétrico do coração é ilustrado.

\begin{figure}[H]
  \centering
  \caption{Sistema de condução do coração}
   \includegraphics[width=0.6\textwidth]{figuras/coracao_sistema_eletrico.png} % insere o tikzpicture puro
  \label{fig:coracao_esquema_eletrico}
    \legend{Fonte: Adaptado de \citeonline{Mitchell_Arritmias_MSD}}
\end{figure}

Segundo \citeonline{Mitchell_Arritmias_MSD}, o batimento cardíaco normal se inicia no nódulo sinusal (1), localizado no átrio direito, que atua como o marcapasso natural do coração. A corrente elétrica propaga-se do átrio direito para o esquerdo (2), promovendo sua contração e o bombeamento do sangue para os ventrículos. Em seguida, o impulso atinge o nódulo atrioventricular (3) — conexão entre os átrios e ventrículos — onde é temporariamente retardado, permitindo que os átrios se contraiam completamente e encham as câmaras inferiores.

Posteriormente, a corrente percorre o feixe de His (4), que se divide e conduz o impulso para ambos os ventrículos (5), promovendo sua contração e o bombeamento do sangue para o restante do corpo.

\section{O eletrocardiograma}
\label{sec:ecg}

O eletrocardiograma, ECG, é um enxame não invasivo usado para medir a atividade elétrica do coração. Ele é feito a partir do contato de eletrodos, chamados de derivações ou \textit{leads}, sobre a pele.
A quantidade de eletrodos varia, mas geralmente são 12 \cite{msd_ecg}.

Ao registrar a magnitude e direção da corrente, as derivações geram uma onda que representa a atividade elétrica do coração. O passo a passo descrito em \ref{sec:funciionamento_coracao} é refletido em sua morfologia.

Na figura \ref{fig:ecg_exemplo_coracao}, é ilustrado um  ECG de um batimento, observe que ele é subdividido em: onda P, complexo QRS e onda T \cite{msd_ecg}. Note que cada uma dessas partes se refere a um estágio do batimento.

\begin{figure}[H]
  \centering
  \caption{Exemplo de ECG com sua morfologia destacada}
   \includegraphics[width=0.8\textwidth]{figuras/ecg_exemplo_coracao.png} % insere o tikzpicture puro
  \label{fig:ecg_exemplo_coracao}
    \legend{Fonte: Adaptado de  \citeonline{msd_ecg}}
\end{figure}

Dentre as doenças que podem ser diagnósticas via ECG estão as arritmias cardíacas. 

\section{Arritmias}


\subsection{Arritmias clínicas}

As arritmias são alterações no ritmo cardíaco que podem ter diversas causas, incluindo alterações hormonais, uso de medicamentos, toxinas (como álcool ou cafeína), anomalias eletrolíticas ou doenças cardíacas.
Segundo \citeonline{Mitchell_Arritmias_MSD}, em adultos em repouso, a frequência cardíaca normal varia entre 60 e 100 batimentos por minuto (bpm). Frequências mais baixas, conhecidas como bradicardia sinusal, são comuns em atletas, crianças pequenas, adolescentes, jovens adultos e durante o sono. Por outro lado, a taquicardia sinusal ocorre quando a frequência se eleva, podendo ser observada durante o esforço físico, doenças, estimulação neural simpática ou emoção intensa.

Variações no ritmo cardíaco são fenômenos fisiológicos normais. Durante a respiração, por exemplo, é comum que a frequência aumente e diminua levemente — comportamento conhecido como arritmia sinusal respiratória. O autor ainda observa que um ritmo cardíaco perfeitamente regular pode indicar patologias no sistema nervoso autônomo, como ocorre em casos de diabetes avançado. Dessa forma, ainda não existe um indicador global e definitivo do que seria um ritmo sinusal considerado saudável.
As arritmias podem ser classificadas de forma simplificada em três tipos principais:

\begin{enumerate}
    \item Taquicardia: frequência excessivamente rápida;
    \item Bradicardia: frequência excessivamente lenta;
    \item Irregular: quando os impulsos percorrem o coração por vias irregulares.
\end{enumerate}

Observe na figura \ref{fig:ecg_exemplo_coracao}, exemplos desses três tipos arrítmicos ilustrados em um ECG.

\subsection{Padrões de classificação para algoritmos (AAMI)}
\label{sub_sec:padroes_arritmias_aami}

A AAMI define cinco classes de arritmia: normal (N), ventricular (V), supraventricular (S), fusão (F) e não classificado (Q) \cite{silva2025,saadatnejad2020}.

As arritmias ventriculares incluem, por exemplo, as contrações prematuras ventriculares (PVCs) e os batimentos de escape. 
O batimento ventricular de escape atua como um mecanismo compensatório, funcionando como um “backup” protetivo do coração quando o marcapasso natural falha temporariamente.
De acordo com \citeonline{Sattar_Premature_2025}, as PVCs são batimentos originários dos ventrículos que podem ocorrer mesmo em indivíduos saudáveis. Sua morfologia é variável, dependendo da origem do impulso, de doenças estruturais ou ainda de uso de medicamentos. Quando frequentes, podem causar fadiga e palpitações, evoluindo para disfunções ventriculares e, em alguns casos, representando a primeira manifestação de cardiopatias estruturais.

\citeonline{msdmanuals_ventricular} cita ainda outras causas potenciais de PVCs, como doenças da artéria coronária (especialmente durante ou após infarto), dilatação ventricular decorrente de insuficiência cardíaca e alterações nas válvulas cardíacas. Em pacientes com doenças estruturais, as PVCs podem progredir para arritmias mais graves, como taquicardia ventricular (VT) e fibrilação ventricular (FV), que representam risco de morte súbita.

\citeonline{mitchell2024afib,mitchell2024vt} descrevem tanto a fibrilação quanto a taquicardia ventricular como arritmias originadas nos ventrículos.

A taquicardia ventricular (TV) produz uma frequência cardíaca de até 120 bpm e é formada por uma sequência de contrações ventriculares prematuras (PVCs). Quando persiste por mais de 30 segundos, recebe a denominação de taquicardia sustentada. Costuma ocorrer em indivíduos com alterações estruturais cardíacas, como infarto do miocárdio, falha cardíaca ou cardiomiopatia. Os sintomas incluem fraqueza, tontura e desconforto torácico. Caso persista por mais de 30 segundos, o tratamento é indicado mesmo na ausência de sintomas, uma vez que pode evoluir para fibrilação ventricular.

Outro tipo de arritmia ventricular é o \textit{flutter} ventricular. Segundo o \citeonline{mesh_ventricular_flutter}, elas são caracterizadas por uma taquicardia extremamente rápida e instável hemodinamicamente (150 a 300 bpm).
São potencialmente fatais e, tipicamente, evoluem para a fibrilação ventricular.

A fibrilação ventricular (FV), por outro lado, é caracterizada por batimentos rápidos e desordenados, resultantes de sinais elétricos caóticos nos ventrículos. Essa condição leva à perda de consciência em poucos segundos e à morte caso não haja intervenção imediata, configurando-se como um tipo de parada cardíaca. Entre suas causas estão afogamentos, choques elétricos e falha cardíaca.

Apesar de ocorrerem também em indivíduos saudáveis, as PVCs possuem significância clínica, pois podem estar associadas a outros tipos de arritmias ventriculares graves. É importante considerar o contexto do batimento: no caso da taquicardia ventricular, por exemplo, o diagnóstico é estabelecido a partir de uma sequência de PVCs que produz uma frequência cardíaca elevada (>20 bpm). Assim, o diagnóstico depende de uma combinação de características temporais e morfológicas, observando-se o contexto dos batimentos e, naturalmente, os sintomas clínicos.

Dentro da classe dos batimentos normais, além do batimento típico descrito na \ref{sec:funciionamento_coracao}, incluem-se também os batimentos atriais de escape e os bloqueios do ramo esquerdo e direito. Estes últimos, embora inofensivos por si só, podem indicar condições cardíacas subjacentes mais graves, como doença da artéria coronária ou infarto do miocárdio prévio \cite{mitchell2024hisbloqueio}.

Segundo o Texas Heart Institute \cite{texasheart_arrhythmias}, arritmias superventricular se originam acima dos ventrículos, como nos 
átrios ou nos caminhos de condução. Geralmente, são mais benignas que as arritmias ventriculares podendo ocorrer, assim como os PVCs, em 
resposta ao consumo de cafeína, tabaco, álcool, tosse ou remédio para resfriado. Outras causas incluem problemas na tireoide. Elas podem causar
palpitações, falta de ar ou aperto no peito.

As contrações superventriculares prematuras, por exemplo, ocorrem quando o átrio se contrai muito cedo. A fibrilação atrial, por outro lado,
são batimentos rápidos e irregulares, causadas por contrações desordenadas das fibras musculares. São as principais causas de 
AVC em idosos, pois causam o acúmulo de sangue nos átrios que podem coagular e viajar até o cérebro. 

\citeonline{statpearls_svt} descreve outros tipos de arritmias superventriculares, como as traquicardias superventricular que são formadas 
por desordens rítmicas rápidas, taquicardia, que são caracterizadas por um complexo QRS mais estreito, menor que 120 ms, e uma 
frequência cardíaca alta. Em adultos, ela é superior a 100 bpm enquanto que em crianças, varia de 180 à 220 bpm. 

Os autores apontam como as traquicardias superventriculares mais comuns a taquicardia atrioventricular nodal reentrante e a taquicardia
atrioventricular recíproca. 

Portanto, as arritmias podem ser classificadas de acordo com sua origem — ventriculares, quando se iniciam nos ventrículos, ou supraventriculares, quando se originam acima deles, como nos átrios.
Essas alterações podem se manifestar por um ritmo cardíaco acelerado ou retardado, ou ainda por uma condução elétrica anormal, o que se reflete na morfologia do traçado eletrocardiográfico (ECG).

Não há, entretanto, padrões universais que definam o que é uma atividade cardíaca normal, pois o ECG apresenta variações fisiológicas relacionadas, por exemplo, à faixa etária, nível de atividade física ou condições clínicas individuais. Diversos fatores podem influenciar a forma do sinal e gerar anomalias que nem sempre têm significado patológico.

Por fim, a diversidade das arritmias também se expressa dentro de uma mesma classe. No caso das arritmias ventriculares, por exemplo, é possível observar desde contrações ventriculares prematuras (PVCs) até taquicardias ventriculares (TVs), ilustrando a amplitude de manifestações possíveis.