\chapter{Fundamentação Teórica}

\section{ECG}

\subsection{O funcionamento do coração}


\subsection{Tipos de arritmia}

As arritmias podem ser classificadas de forma simplificada em três tipos principais:
\begin{itemize}
    \item Taquicardia: frequência excessivamente rápida;
    \item Bradicardia: frequência excessivamente lenta;
    \item Irregular: quando os impulsos percorrem o coração por vias irregulares.
\end{itemize}

\subsection{O batimento cardíaco}

O batimento cardíaco inicia-se no nódulo sinoatrial, cuja corrente elétrica atravessa o átrio direito e, em seguida, o átrio esquerdo, promovendo sua contração. O sangue é então impulsionado dos átrios para os ventrículos. A corrente elétrica passa pelo nódulo atrioventricular, único ponto de conexão entre átrios e ventrículos, que retarda o impulso, garantindo enchimento completo dos ventrículos. 

Em seguida, o impulso segue pelo feixe de His, que se divide em ramos para conduzir a corrente a cada ventrículo, permitindo sua ativação uniforme e subsequente contração, bombeando o sangue para o corpo \cite{msd_ecg}.

\subsection{O que é ECG?}

Segundo Cascino e Shea \cite{msd_ecg}, o eletrocardiograma (ECG) é um exame não invasivo que registra a atividade elétrica do coração. 
Ele é realizado pela colocação de eletrodos na pele do paciente, geralmente 12, chamados de derivações. 
Esses eletrodos registram tanto a direção quanto a magnitude da corrente elétrica. 

O registro resultante gera uma onda que reflete a atividade elétrica do coração. 
Cada etapa do ciclo cardíaco é representada na morfologia do traçado: 
a onda \textbf{P} corresponde à ativação dos átrios, 
o complexo \textbf{QRS} à ativação dos ventrículos 
e a onda \textbf{T} ao processo de repolarização ventricular.  

O ECG é uma ferramenta fundamental no diagnóstico de problemas cardíacos, 
permitindo identificar, por exemplo, episódios de infarto do miocárdio, 
oferta insuficiente de sangue e oxigênio ao coração (isquemia), 
hipertrofia das paredes cardíacas e diferentes tipos de arritmias.