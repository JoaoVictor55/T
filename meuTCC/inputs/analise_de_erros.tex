\chapter{Análise de erros no pior \textit{fold}}
\label{ch:analise_erros_pior_fold}

Pelos critérios adotados, o terceiro fold foi o de pior desempenho em ambos os modelos.
Como o recall foi superior à precisão, supõe-se que a causa esteja relacionada à presença de batimentos normais com características morfológicas atípicas, o que pode ter confundido os modelos.
Para fins de ilustração, apresenta-se a seguir uma breve análise de erros do desempenho do melhor modelo em seu cenário mais desafiador.
Devido a essa limitação, os resultados discutidos não permitem conclusões generalizáveis, servindo apenas como apoio à interpretação dos achados.

\section{Análise de erros do modelo híbrido CNN com GRU}
\label{sec:analise_erros_cnn_gru}

Na tabela \ref{tab:erros_acertos_por_paciente} a seguir, é possível ver que a maioria dos erros foi oriunda de um paciente, o 203.

\begin{table}[H]
\centering
\caption{Total dos erros e acertos por paciente no \textit{fold} de validação}
\label{tab:erros_acertos_por_paciente}
\begin{tabular}{lcc}
\hline
\textbf{Pacientes} & \textbf{Erros} & \textbf{Acertos}\\
\hline
119 & 0 &  1972 \\
203 & 772  & 2186\\
205 & 11 & 2616\\
209 & 1 & 2606\\
\hline
\end{tabular}
\legend{Fonte: Elaborado pelo autor.}
\end{table}

Aproximadamente, 98,46\% de todos os erros foram desse paciente. O modelo errou em torno de 35,31\% de seus batimentos. Conforme visto na 
figura \ref{fig:matriz_confusao_cnn_gru_pior_fold}, a maioria desses erros são de falsos positivos.

No paciente 209, o modelo acertou a única classe positiva que existia. O único erro cometido foi um falso positivo. Já no paciente 
205, o modelo acertou 69 das 71 classes positivas e errou 9 classes negativas, das 2.556. Nesses dois pacientes, a classe positiva 
era extremamente rara, mas em números absolutos, a maioria dos erros foram de falsos positivos.

Segundo as anotações do MIT-BIH, disponíveis em \cite{physionet_annotations}, o paciente 203 é considerado como muito difícil. As anotações ainda citam
a presença de mudança de morfologia no complexo QRS e contrações ventriculares prematuras (PVC) de múltiplas formas.

Na figura \ref{fig:matriz_confusao_paciente_mais_dificil}, é mostrada a matriz de confusão desse paciente.

\begin{figure}[H]
  \centering
  \caption{Matriz de confusão do paciente 203}
   \includegraphics[width=0.7\textwidth]{figuras/analise_erros/matriz_confusao_paciente_mais_dificil.png} 
  \label{fig:matriz_confusao_paciente_mais_dificil}
  \legend{Fonte: Elaborado pelo autor.}
\end{figure}

O modelo confundiu seis batimentos ventriculares como normais e 766 normais como ventriculares. Na figura
\ref{fig:erro_acert_neg_class_paciente_mais_dificil}, é ilustrado duas sequencias desse paciente, na primeira
uma sequencia normal classificada como arrítmica e na segunda uma normal corretamente classificada.

\begin{figure}[H]
  \centering
  \caption{ECG normal do paciente 203}
   \includegraphics[width=1.0\textwidth]{figuras/analise_erros/ecg_erro_acerto_neg_paciente_mais_dificil.png} 
  \label{fig:erro_acert_neg_class_paciente_mais_dificil}
  \legend{Fonte: Elaborado pelo autor.}
\end{figure}

É possível observar a forte presença de ruído em ambos os casos. E a presença de batimentos com a morfologia 
bem deformada; após a amostra 2000 no primeiro gráfico e após a amostra 1000 no segundo. 

O modelo previu uma probabilidade de aproximadamente 0,97, indicando não só um erro, mas um erro com muita confiança. Já no segundo 
ECG, a probabilidade prevista foi de aproximadamente 0,12. Foi um acerto relativamente confiante.

Para comparação, na figura \ref{fig:acert_neg_class_paciente_mais_facil}, paciente para o qual o modelo não cometeu erros,
abaixo é ilustrado uma sequência normal.

\begin{figure}[H]
  \centering
  \caption{ECG normal do paciente 119.}
   \includegraphics[width=1.0\textwidth]{figuras/analise_erros/ecg_sequencia_normal_neg_paciente_mais_facil.png} 
  \label{fig:acert_neg_class_paciente_mais_facil}
  \legend{Fonte: Elaborado pelo autor.}
\end{figure}

É possível notar uma sequencia mais limpa e com o complexo QRS com morfologia usual. Note em torno da amostra
2000 uma contração prematura ventricular.

Na figura \ref{fig:erro_acert_pos_class_paciente_mais_dificil} é ilustrado duas sequências arrítmicas do paciente 203, a primeiro o modelo acertou e a segunda ele errou:

Em ambos os casos, é observável o ruído presenta na figura \ref{fig:erro_acert_neg_class_paciente_mais_dificil}. O 
último batimento da sequência também apresenta uma morfologia diferente da usual.

\begin{figure}[H]
  \centering
  \caption{ECG normal do paciente 203: acerto e erro}
   \includegraphics[width=1.0\textwidth]{figuras/analise_erros/ecg_erro_acerto_pos_paciente_mais_dificil.png} 
  \label{fig:erro_acert_pos_class_paciente_mais_dificil}
  \legend{Fonte: Elaborado pelo autor.}
\end{figure}

A probabilidade prevista para o caso errado foi de aproximadamente 16\%, indicando um erro com confiança. Já o acerto 
teve uma probabilidade de 0,97\%. A ausência da arritmia usual pode ser uma causa, o modelo pode ter associado a sua 
presença a classe ser arrítmica. Porém, novamente, como o modelo é caixa preta, não é possível afirmar.

Já na figura \ref{fig:acert_posclass_paciente_mais_facil}, é ilustrada uma sequencia arrítmica do paciente 119.
Observe no último batimento, uma arritmia ventricular.

\begin{figure}[H]
  \centering
  \caption{ECG arrítmico do paciente 119.}
   \includegraphics[width=1.0\textwidth]{figuras/analise_erros/ecg_sequencia_normal_pos_paciente_mais_facil.png} 
  \label{fig:acert_posclass_paciente_mais_facil}
  \legend{Fonte: Elaborado pelo autor.}
\end{figure}

Para visualizar o quão confiante o modelo foi em seus erros, abaixo está a curva de calibração para esse paciente:

\begin{figure}[H]
  \centering
  \caption{Curva de calibração para o paciente 203.}
   \includegraphics[width=1.0\textwidth]{figuras/analise_erros/curva_calibracao_pior_paciente.png} 
  \label{fig:curva_calibracao_pior_paciente}
  \legend{Fonte: Elaborado pelo autor.}
\end{figure}

Pela curva, observa-se um cenário de superconfiança. Quando o modelo prever, por exemplo, que em média, 
90\% dos batimentos são arrítmicos, menos de 60\% deles realmente são.

Apesar das diferenças naturais entre ECGs de pacientes diferentes, pelo o que foi visto no gráfico e o que está registrado
nas anotações do MIT-BIH; o ECG desse paciente possui muito ruído e QRS com morfologia mais diferente, especialmente
quando comparando com as do paciente 119; essas diferenças pode ter levado o modelo ao erro. Apesar de que, por ser 
pouca interpretabilidade, não é possível fazer uma afirmação.

Além disso, observa-se um alta confiança nos erros. Novamente, a maioria dos erros são de falsos positivos, que são menos 
danosos que um falso negativo no contexto médico. Entretanto, a superconfiança do modelo adiciona uma outra dimensão ao problema; 
quando um modelo é bem calibrado, então a média da probabilidade prevista para um grupo corresponde a média da frequência da classe positiva 
nesse grupo; o que ajuda muito na tomada de decisão, mesmo que a performance do modelo não seja boa.

O último ponto é a interpretabilidade, que não foi tratada nesse trabalho, que permite ao tomador de decisão entender o que numa determinada amosta mais influenciou 
o resultado do modelo. Isso adiciona mais transparência e certamente, torna a confiança do mesmo no modelo maior. Conforme 
ilustrado nessa seção, o fato de redes neurais terem pouca interpretabilidade dificulta entender o porquê a rede errou ou acertou.


