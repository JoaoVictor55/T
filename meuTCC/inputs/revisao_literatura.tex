\chapter{Trabalhos correlatos}

\citeonline{chazal2004} propuseram um método automático para classificação de arritmias nas cinco classes definidas pela AAMI (descritas na seção \ref{sub_sec:padroes_arritmias_aami}), utilizando o banco de dados MIT-BIH. O classificador empregado foi o discriminante linear, um método estatístico paramétrico baseado na suposição de distribuição Gaussiana dos dados.

A metodologia incluiu etapas de pré-processamento, extração de features e classificação. Foram avaliadas diferentes configurações e estratégias de particionamento, incluindo a tradicional (intrapaciente) e a proposta pelos autores (interpaciente, com os conjuntos Ds1 e Ds2 descritos em \ref{sec:particionamento}). O estudo demonstrou que a partição intrapaciente gera resultados otimistas e não representa adequadamente a capacidade de generalização do modelo.

O pré-processamento aplicava filtros de mediana e passa-baixa para remoção do baseline wander. Em seguida, as features foram extraídas a partir de duas abordagens principais: (i) janelas fixas centradas no pico R e (ii) janelas adaptadas à duração do batimento, considerando o complexo QRS e a onda T. Também foram utilizadas medidas temporais, como duração do QRS, intervalos RR e presença da onda P.

A melhor configuração encontrada combinava features segmentadas com janela temporal flexível, intervalos RR, presença da onda P e durações do QRS e da onda T — todas extraídas sem escalar o sinal —, processadas por dois discriminantes lineares independentes, um para cada derivação do MIT-BIH. Essa abordagem obteve desempenho superior às demais e se tornou referência para estudos posteriores.

Apesar dos bons resultados, o método depende fortemente da qualidade das features e do pré-processamento, sendo sensível à variação morfológica entre pacientes — um ponto que trabalhos mais recentes, baseados em aprendizado profundo, buscam superar.

Em \citeonline{saadatnejad2020}, foi proposto um modelo baseado em redes LSTM (ver Seção~\ref{sec:ann}) para a classificação de arritmias voltado a dispositivos móveis. 
Com o objetivo de atender às restrições computacionais de sistemas embarcados, os autores desenvolveram uma arquitetura composta por dois modelos simples que realizam 
classificações independentes, seguidos por uma rede neural artificial (ANN) responsável pela decisão final.

A entrada do modelo consiste em batimentos cardíacos segmentados utilizando uma janela temporal fixa, 
características extraídas via transformada \textit{wavelet} de Daubechies de ordem dois, e intervalos RR, conforme sugerido por \citeonline{chazal2004}. 
O sistema é dividido em dois submodelos, denominados \(\alpha\) e \(\beta\). 
O primeiro (\(\alpha\)) possui dois ramos de LSTM: um recebe o sinal reamostrado (\textit{downsampled}) concatenado com os intervalos RR, 
enquanto o outro recebe o sinal combinado com as características \textit{wavelet}. 
Já o segundo submodelo (\(\beta\)) é composto por duas camadas LSTM, recebendo como entrada os componentes principais do sinal, 
concatenados com as características \textit{wavelet} e os intervalos RR. 
Em ambos os casos, a saída das LSTM é conectada a uma camada totalmente conectada (FC, \textit{Fully Connected}) para a classificação.

O treinamento foi conduzido sob uma divisão \textit{intrapaciente}, na qual foi desenvolvido um modelo especialista para cada indivíduo. 
De acordo com as recomendações da AAMI, os autores separaram os dados em conjuntos locais e globais: 
os dados locais corresponderam aos cinco primeiros minutos dos registros dos pacientes do conjunto Ds2, 
enquanto o conjunto global foi composto pelo conjunto Ds1 completo. 
Para teste, utilizou-se o restante do conjunto Ds2, excluindo os minutos empregados no treinamento.

Por se tratar de um modelo voltado a dispositivos vestíveis — e, portanto, de uso pessoal —, essa divisão não constitui necessariamente um vazamento de dados, 
embora o viés de sobreposição entre indivíduos permaneça. Assim, é importante considerar esse aspecto ao comparar os resultados com trabalhos que adotam a divisão \textit{interpaciente}.
