\chapter{Análise de erros no pior \textit{fold}}
\label{ch:analise_erros_pior_fold}

Nesta seção, será apresentada uma breve análise de erros do modelo híbrido em seu pior \textit{fold}. 
Conforme os critérios adotados, o eleito foi o terceiro.

Como o recall foi superior à precisão, foi suposto que a causa esteja relacionada à presença de batimentos normais com características morfológicas atípicas, o que pode ter confundido os modelos.
Assim o objetivo dessa seção é fazer uma breve exploração desse \textit{fold} e verificar como os erros estão distribuídos em relação aos pacientes,
e quais características os dados desses pacientes têm; como a morfologia do ECG e o intervalo RR.
Por fim, será apresentado também a curva de calibração do modelo.

Devido ao fato de redes neurais serem modelos caixa-preta, não é possível afirmar quais características contribuíram para 
os erros, logo, a análise serve apenas para construir uma intuição inicial para os resultados achados.

\section{Análise de erros do modelo híbrido CNN com GRU}
\label{sec:analise_erros_cnn_gru}

Na Tabela~\ref{tab:erros_acertos_por_paciente}, a seguir, é possível ver que a maioria dos erros foi oriunda de um paciente, o 203.

\begin{table}[H]
\centering
\caption{Total dos erros e acertos por paciente no \textit{fold} de validação}
\label{tab:erros_acertos_por_paciente}
\begin{tabular}{lcc}
\hline
\textbf{Pacientes} & \textbf{Erros} & \textbf{Acertos}\\
\hline
119 & 0 &  1972 \\
203 & 772  & 2186\\
205 & 11 & 2616\\
209 & 1 & 2606\\
\hline
\end{tabular}
\legend{Fonte: Elaborado pelo autor.}
\end{table}

Aproximadamente, 98,46\% de todos os erros foram desse paciente. Conforme visto na 
Figura~\ref{fig:matriz_confusao_cnn_gru_pior_fold}, a maioria desses erros são de falsos positivos.
Segundo as anotações do MIT-BIH, disponíveis em \cite{physionet_annotations}, o paciente 203 é considerado como muito difícil. As anotações ainda citam
a presença de mudança de morfologia no complexo QRS e contrações ventriculares prematuras (PVC) de múltiplas formas.

Já no paciente 209, o modelo acertou a única classe positiva que existia. O único erro cometido foi um falso positivo. No paciente 
205, o modelo acertou 69 das 71 classes positivas e errou 9 classes negativas, das 2.556. Nesses dois pacientes, a classe positiva 
era extremamente rara, mas em números absolutos, a maioria dos erros foram de falsos positivos.

Como o paciente 203 dominou os erros neste \textit{fold} e para o paciente 119, não houve erros, na próxima seção, a análise 
se concentrar neste dois casos.

\subsubsection{Comparação morfológica entre o paciente 203 e 119}

Na Figura~\ref{fig:matriz_confusao_paciente_mais_dificil}, é mostrada a matriz de confusão desse paciente.

\begin{figure}[H]
  \centering
  \caption{Matriz de confusão do paciente 203}
   \includegraphics[width=0.7\textwidth]{figuras/analise_erros/matriz_confusao_paciente_mais_dificil.png} 
  \label{fig:matriz_confusao_paciente_mais_dificil}
  \legend{Fonte: Elaborado pelo autor.}
\end{figure}

Das 2.516 sequências normais, o modelo classificou 766 como arrítmicas e acertou 1.750 e das 472 sequências arrítmicas, o modelo acertou 
436 e errou 6. Assim, ao todo, o modelo errou 35,31\% dos batimentos do paciente 209;

Na Figura~\ref{fig:erro_acert_neg_class_paciente_mais_dificil}, é ilustrado duas sequencias desse paciente, na primeira
um falso positivo e na segunda um verdadeiro negativo.

\begin{figure}[H]
  \centering
  \caption{ECG normal do paciente 203}
   \includegraphics[width=1.0\textwidth]{figuras/analise_erros/ecg_erro_acerto_neg_paciente_mais_dificil.png} 
  \label{fig:erro_acert_neg_class_paciente_mais_dificil}
  \legend{Fonte: Elaborado pelo autor.}
\end{figure}

É possível observar a forte presença de ruído em ambos os casos. E a presença de batimentos com a morfologia 
bem deformada; após a amostra 2000 no primeiro gráfico e após a amostra 1000 no segundo. 

O modelo tinha 97\% de confiança que o primeiro ECG era da classe positiva, logo foi um erro com 
muita confiança. No segundo caso, o modelo tinha 12\% de confiança que a 
sequência pertencia a classe positiva, ou seja, 88\% de chance de ser da classe negativa; um acerto com confiança.

Na Figura~\ref{fig:acert_neg_class_paciente_mais_facil}, é ilustrada a sequência normal do paciente mais fácil;

\begin{figure}[H]
  \centering
  \caption{ECG normal do paciente 119.}
   \includegraphics[width=0.82\textwidth]{figuras/analise_erros/ecg_sequencia_normal_neg_paciente_mais_facil.png} 
  \label{fig:acert_neg_class_paciente_mais_facil}
  \legend{Fonte: Elaborado pelo autor.}
\end{figure}

É possível notar uma sequencia mais limpa e com o complexo QRS com morfologia usual. Note em torno da amostra
2000 uma contração prematura ventricular.

Na Figura \ref{fig:erro_acert_pos_class_paciente_mais_dificil} é ilustrado duas sequências arrítmicas do paciente 203, a primeiro o modelo errou e a segunda ele acertou:

Em ambos os casos, é observável o ruído presenta na Figura~\ref{fig:erro_acert_neg_class_paciente_mais_dificil}. O 
último batimento da sequência também apresenta uma morfologia diferente da usual.

\begin{figure}[H]
  \centering
  \caption{ECG arrítmico do paciente 203: acerto e erro}
   \includegraphics[width=1.0\textwidth]{figuras/analise_erros/ecg_erro_acerto_pos_paciente_mais_dificil.png} 
  \label{fig:erro_acert_pos_class_paciente_mais_dificil}
  \legend{Fonte: Elaborado pelo autor.}
\end{figure}

O modelo tinha 16\% de confiança que era um caso da classe positiva, logo, 84\% que era da classe negativa; um erro 
confiante. É possível notar uma morfologia relativamente mais limpa e uniforme, principalmente quando comparada às demais sequências 
apresentadas deste paciente, levando o modelo a confundi-la com uma sequência normal.

Já no segundo caso, o modelo tinha 97\% de confiança que era um caso da classe positiva; sendo um 
acerto com confiança. Nota-se a presença de arritmias ventriculares e um sinal com mais ruído.

Na Figura~\ref{fig:acert_posclass_paciente_mais_facil}, é ilustrada uma sequencia arrítmica do paciente 119.
Observe no último batimento, uma arritmia ventricular.

\begin{figure}[H]
  \centering
  \caption{ECG arrítmico do paciente 119.}
   \includegraphics[width=0.82\textwidth]{figuras/analise_erros/ecg_sequencia_normal_pos_paciente_mais_facil.png} 
  \label{fig:acert_posclass_paciente_mais_facil}
  \legend{Fonte: Elaborado pelo autor.}
\end{figure}

\subsubsection{Comparação temporal entre o paciente 203 e 119}

Na Figura~\ref{fig:poincarefp}, é ilustrado o gráfico de dispersão de um falso positivo do paciente 209.
Note o último batimento da sequência (o 16º), em relação aos demais, ele ocorreu de forma mais prematura uma vez que se encontra na
região inferior esquerda do gráfico, em comparação a outros batimentos — como o quarto, o oitavo e o décimo. Além disso, o intervalo
entre ele e seu sucessor é menor do que o intervalo em relação ao batimento anterior, o que poderia sugerir um padrão compatível com
um batimento ventricular prematuro (PVC). No entanto, essa interpretação dependeria da frequência cardíaca do paciente, e como
discutido na seção \ref{sub_sec:padroes_arritmias_aami}, não existe um padrão universal de ECG normal.

A Figura~\ref{fig:poincaretp} mostra um caso de verdadeiro positivo do mesmo paciente. O último batimento também ocorre de forma antecipada, porém sua posição mais próxima à linha de identidade indica que o intervalo com o sucessor é aproximadamente igual ao intervalo com o anterior.

\begin{figure}[h!]
    \centering
    \caption{Scatter plot do paciente 203 de um falso positivo e um verdadeiro positivo.}

    \begin{minipage}{0.9\textwidth}
        \centering
        \includegraphics[width=0.7\linewidth]{figuras/analise_erros/poincare_paciente_mais_dificil_fp.png}
        \subcaption{Falso positivo do paciente 203}
        \label{fig:poincarefp}
    \end{minipage}

    %\vspace{0.5cm} % espaço entre as imagens

    \begin{minipage}{0.9\textwidth}
        \centering
        \includegraphics[width=0.7\linewidth]{figuras/analise_erros/poincare_paciente_mais_dificil_tp.png} 
        \subcaption{Verdadeiro positivo do paciente 203}
        \label{fig:poincaretp}
    \end{minipage}

    \legend{Fonte: Elaborado pelo autor.}
    \label{fig:poincare_fp_tp_maisDificil}
\end{figure}


\iffalse
\begin{figure}[H]
  \centering
  \caption{Scatter plot do paciente 203 de um falso positivo.}
   \includegraphics[width=0.70\textwidth]{figuras/analise_erros/poincare_paciente_mais_dificil_fp.png} 
  \label{fig:poicare_fp}
  \legend{Fonte: Elaborado pelo autor.}
\end{figure}

\begin{figure}[H]
  \centering
  \caption{Scatter plot do paciente 203 de um verdadeiro positivo.}
   \includegraphics[width=0.7\textwidth]{figuras/analise_erros/poincare_paciente_mais_dificil_tp.png} 
  \label{fig:poicare_tp}
  \legend{Fonte: Elaborado pelo autor.}
\end{figure}
\fi

Já a Figura~\ref{fig:poincare_tn_MaisDificil} apresenta um verdadeiro negativo. Embora os pontos estejam igualmente dispersos, o último batimento aparece mais deslocado para o canto inferior direito quando comparado aos dois casos anteriores. Essa diferença pode ter contribuído para que o modelo confundisse o caso negativo anterior como um batimento prematuro, resultando em um falso positivo.

Por fim, a Figura~\ref{fig:poincare_tn_MaisFacil} mostra uma sequência normal do paciente 119. Diferentemente das anteriores, os batimentos formam agrupamentos mais concentrados, com o último batimento situado próximo ao centro e à linha de identidade.

\iffalse
\begin{figure}[H]
  \centering
  \caption{Scatter plot do paciente 203 de um verdadeiro negativo.}
   \includegraphics[width=0.75\textwidth]{figuras/analise_erros/poincare_paciente_mais_dificil_tn.png} 
  \label{fig:poicare_tn}
  \legend{Fonte: Elaborado pelo autor.}
\end{figure}

\begin{figure}[H]
  \centering
  \caption{Scatter plot do paciente 119 de uma sequência normal.}
   \includegraphics[width=0.7\textwidth]{figuras/analise_erros/poincare_paciente_mais_facil_neg.png} 
  \label{fig:poicare_neg_maisfacil}
  \legend{Fonte: Elaborado pelo autor.}
\end{figure}
\fi

\begin{figure}[h!]
    \centering
    \caption{Scatter plot da sequência normal do paciente 203 e 119}

    \begin{minipage}{0.9\textwidth}
        \centering
        \includegraphics[width=0.7\linewidth]{figuras/analise_erros/poincare_paciente_mais_dificil_tn.png}
        \subcaption{Verdadeiro negativo do paciente 203}
        \label{fig:poincare_tn_MaisDificil}
    \end{minipage}

    %\vspace{0.5cm} % espaço entre as imagens

    \begin{minipage}{0.9\textwidth}
        \centering
        \includegraphics[width=0.7\linewidth]{figuras/analise_erros/poincare_paciente_mais_facil_neg.png} 
        \subcaption{Verdadeiro negativo do paciente 119}
        \label{fig:poincare_tn_MaisFacil}
    \end{minipage}

    \legend{Fonte: Elaborado pelo autor.}
    \label{fig:poincare_normal_maisdificil_maisfacil}
\end{figure}

Para visualizar o quão confiante o modelo foi em seus erros, foi feito a curva de calibração, na Figura~\ref{fig:curva_calibracao_pior_paciente}, do modelo para o paciente 203 e o paciente 119.

\begin{figure}[h]
  \centering
  \caption{Curva de calibração para o paciente 203 e 119.}
   \includegraphics[width=0.9\textwidth]{figuras/analise_erros/curva_calibracao_pior_paciente_e_melhor_paciente.png} 
  \label{fig:curva_calibracao_pior_paciente}
  \legend{Fonte: Elaborado pelo autor.}
\end{figure}

A análise de calibração indica que o modelo está superconfiante em ambos os pacientes. Este cenário é caracterizado pelas curvas de calibração que se mantêm abaixo da diagonal principal.

No caso do paciente 119, a falta de calibração é menos sevara, pois sua curva se aproxima mais da diagonal principal. Já para o paciente 203, a superconfiança é mais intensa, especialmente 
em \textit{bins} abaixo de 80\%. Um exemplo é quando o modelo prevê que 90\% de um grupo será arrítmico, a realidade é que menos de 60\% deles o são.

Conforme discutido na seção \ref{sec:metricas}, a calibração não está diretamente relacionada ao desempenho, tanto que em ambos os pacientes, a rede foi superconfiante. Seria possível 
que o modelo tivesse um desempenho baixo e fosse bem calibrado; neste cenário, o tomador de decisão poderia confiar que as probabilidades
previstas seriam realistas, mesmo que o modelo não discernisse tão bem.

\iffalse
Em ambos os casos, o modelo foi superconfiante, pois as curvas se mantém a baixo da diagonal principal. Entretanto,
como a curva do paciente 119 se aproxima mais da diagonal, então a para este paciente, a falta de calibração
foi menos severa. Após o \textit{bin} de probabilidade média de 80\%, próximo ao de 100\%, a curva do paciente 119 cruza a diagonal; assim 
o cenário se inverte, o modelo passa a ser pouco confiante.

Para o paciente 203, considerando que em um grupo, o modelo preveja que em média, 90\% deles serão arrítmicos, a realidade é que menos de 60\% deles são. É possível observar,
também, como a superconfiança é mais intensa para \textit{bins} abaixo dos 80\%.

Conforme a discussão da seção \ref{sec:metricas}, a calibração não é diretamente relacionada ao desempenho de um modelo medida pelas métricas usuais.
Isto pode ser observada na curva de calibração do paciente 113 onde o desempenho foi alto, mas a falta de calibração também existe.
Assim, mesmo se o modelo fosse muito sensível e tivesse uma precisão baixa, caso ele fosse bem calibrado, o tomador de decisão poderia 
confiar que as probabilidades previstas seriam mais realista.
\fi

\subsubsection{Conclusão da análise de erros}

O objetivo desta seção foi analisar as sequências de batimentos do modelo híbrido em seu pior \textit{fold} para ganhar 
uma intuição adicional às razões da falha do mesmo. Foi identificado que a grande maioria dos erros veio de um único 
paciente, o 203, que as anotações do MIT-BIH o classifica como muito difícil, devido a presença de ruído e arritmias 
atípicas. 

Comparando o paciente 203 com o 119 — paciente para o qual o modelo não cometeu erros — foi identificado que esse possui 
um sinal mais limpo e sem grandes morfologias atípicas. Além disso, no paciente 203, 
num caso de falso positivo, o intervalo do último batimento da 
em relação ao seu antecessor é mais curto, o que poderia fazer com que o modelo o confundisse com uma arritmia. 

Além disso, a curva de calibração mostrou um cenário de superconfiança para ambos os pacientes, reforçando a ideia de que 
ela não tem relação direta com o desempenho.

Conforme visto na seção \ref{ch:resultados}, este \textit{fold} também foi o pior caso do modelo GRU e o mesmo ainda se saiu pior que o 
híbrido, com menor \textit{recall} e precisão. O que pode ser outra evidência da vantagem da CNN. Como a CNN atua como extrator de 
\textit{features} da morfologia, ela pode ter mitigado o impacto do ruído e da forma menos usual do ECG deste paciente, resultando 
em um desempenho melhor. Além disso, conforme visto, é possível que os intervalos RR tenha contribuído menos para a separação. Como ambos
os modelos receberam as mesmas \textit{features} RR, então a vantagem da extração feita pela CNN, pode ter ajudado o híbrido a se destacar.
