\chapter{Análise de erros no pior \textit{fold}}
\label{ch:analise_erros_pior_fold}

Pelos critérios adotados, o terceiro fold foi o de pior desempenho em ambos os modelos.
Como o recall foi superior à precisão, supõe-se que a causa esteja relacionada à presença de batimentos normais com características morfológicas atípicas, o que pode ter confundido os modelos.
Para fins de ilustração, apresenta-se a seguir uma breve análise de erros do desempenho do melhor modelo em seu cenário mais desafiador.
Devido a essa limitação, os resultados discutidos não permitem conclusões generalizáveis, servindo apenas como apoio à interpretação dos achados.

\section{Análise de erros do modelo híbrido CNN com GRU}
\label{sec:analise_erros_cnn_gru}

Na tabela \ref{tab:erros_acertos_por_paciente} a seguir, é possível ver que a maioria dos erros foi oriunda de um paciente, o 203.

\begin{table}[H]
\centering
\caption{Total dos erros e acertos por paciente no \textit{fold} de validação}
\label{tab:erros_acertos_por_paciente}
\begin{tabular}{lcc}
\hline
\textbf{Pacientes} & \textbf{Erros} & \textbf{Acertos}\\
\hline
119 & 0 &  1972 \\
203 & 772  & 2186\\
205 & 11 & 2616\\
209 & 1 & 2606\\
\hline
\end{tabular}
\legend{Fonte: Elaborado pelo autor.}
\end{table}

Aproximadamente, 98,46\% de todos os erros foram desse paciente. O modelo errou em torno de 35,31\% de seus batimentos. Conforme visto na 
figura \ref{fig:matriz_confusao_cnn_gru_pior_fold}, a maioria desses erros são de falsos positivos.

No paciente 209, o modelo acertou a única classe positiva que existia. O único erro cometido foi um falso positivo. Já no paciente 
205, o modelo acertou 69 das 71 classes positivas e errou 9 classes negativas, das 2.556. Nesses dois pacientes, a classe positiva 
era extremamente rara, mas em números absolutos, a maioria dos erros foram de falsos positivos.

Segundo as anotações do MIT-BIH, disponíveis em \cite{physionet_annotations}, o paciente 203 é considerado como muito difícil. As anotações ainda citam
a presença de mudança de morfologia no complexo QRS e contrações ventriculares prematuras (PVC) de múltiplas formas.

Na figura \ref{fig:matriz_confusao_paciente_mais_dificil}, é mostrada a matriz de confusão desse paciente.

\begin{figure}[H]
  \centering
  \caption{Matriz de confusão do paciente 203}
   \includegraphics[width=0.7\textwidth]{figuras/analise_erros/matriz_confusao_paciente_mais_dificil.png} 
  \label{fig:matriz_confusao_paciente_mais_dificil}
  \legend{Fonte: Elaborado pelo autor.}
\end{figure}

O modelo confundiu seis batimentos ventriculares como normais e 766 normais como ventriculares. Na figura
\ref{fig:erro_acert_neg_class_paciente_mais_dificil}, é ilustrado duas sequencias desse paciente, na primeira
uma sequencia normal classificada como arrítmica e na segunda uma normal corretamente classificada.

\begin{figure}[H]
  \centering
  \caption{ECG normal do paciente 203}
   \includegraphics[width=1.0\textwidth]{figuras/analise_erros/ecg_erro_acerto_neg_paciente_mais_dificil.png} 
  \label{fig:erro_acert_neg_class_paciente_mais_dificil}
  \legend{Fonte: Elaborado pelo autor.}
\end{figure}

É possível observar a forte presença de ruído em ambos os casos. E a presença de batimentos com a morfologia 
bem deformada; após a amostra 2000 no primeiro gráfico e após a amostra 1000 no segundo. 

O modelo previu uma probabilidade de aproximadamente 0,97, indicando não só um erro, mas um erro com muita confiança. Já no segundo 
ECG, a probabilidade prevista foi de aproximadamente 0,12. Foi um acerto relativamente confiante.

Para comparação, na figura \ref{fig:acert_neg_class_paciente_mais_facil}, paciente para o qual o modelo não cometeu erros,
abaixo é ilustrado uma sequência normal.

\begin{figure}[H]
  \centering
  \caption{ECG normal do paciente 119.}
   \includegraphics[width=1.0\textwidth]{figuras/analise_erros/ecg_sequencia_normal_neg_paciente_mais_facil.png} 
  \label{fig:acert_neg_class_paciente_mais_facil}
  \legend{Fonte: Elaborado pelo autor.}
\end{figure}

É possível notar uma sequencia mais limpa e com o complexo QRS com morfologia usual. Note em torno da amostra
2000 uma contração prematura ventricular.

Na figura \ref{fig:erro_acert_pos_class_paciente_mais_dificil} é ilustrado duas sequências arrítmicas do paciente 203, a primeiro o modelo acertou e a segunda ele errou:

Em ambos os casos, é observável o ruído presenta na figura \ref{fig:erro_acert_neg_class_paciente_mais_dificil}. O 
último batimento da sequência também apresenta uma morfologia diferente da usual.

\begin{figure}[H]
  \centering
  \caption{ECG normal do paciente 203: acerto e erro}
   \includegraphics[width=1.0\textwidth]{figuras/analise_erros/ecg_erro_acerto_pos_paciente_mais_dificil.png} 
  \label{fig:erro_acert_pos_class_paciente_mais_dificil}
  \legend{Fonte: Elaborado pelo autor.}
\end{figure}

A probabilidade prevista para o caso errado foi de aproximadamente 16\%, indicando um erro com confiança. Já o acerto 
teve uma probabilidade de 0,97\%. A ausência da arritmia usual pode ser uma causa, o modelo pode ter associado a sua 
presença a classe ser arrítmica. Porém, novamente, como o modelo é caixa preta, não é possível afirmar.

Já na figura \ref{fig:acert_posclass_paciente_mais_facil}, é ilustrada uma sequencia arrítmica do paciente 119.
Observe no último batimento, uma arritmia ventricular.

\begin{figure}[H]
  \centering
  \caption{ECG arrítmico do paciente 119.}
   \includegraphics[width=1.0\textwidth]{figuras/analise_erros/ecg_sequencia_normal_pos_paciente_mais_facil.png} 
  \label{fig:acert_posclass_paciente_mais_facil}
  \legend{Fonte: Elaborado pelo autor.}
\end{figure}

Para visualizar a distribuição dos intervalos RR, foi utilizado um scatter plot das features RR, em que o eixo X representa o intervalo RR anterior e o eixo Y, o subsequente. 

Na figura \ref{fig:poicare_fp}, é ilustrado um caso de falso positivo referente ao paciente mais desafiador. Observa-se que o último batimento da sequência (o 16º) ocorreu de forma relativamente mais prematura, uma vez que se encontra na região inferior esquerda do gráfico, em comparação a outros batimentos — como o quarto, o oitavo e o décimo. Além disso, o intervalo entre ele e seu sucessor é menor do que o intervalo em relação ao batimento anterior, o que poderia sugerir um padrão compatível com um batimento ventricular prematuro (PVC). No entanto, essa interpretação dependeria da frequência cardíaca do paciente, e como discutido na seção \ref{sub_sec:padroes_arritmias_aami}, não existe um padrão universal de ECG normal.

A figura \ref{fig:poicare_tp} mostra um caso de verdadeiro positivo do mesmo paciente. O último batimento também ocorre de forma antecipada, porém sua posição mais próxima à linha de identidade indica que o intervalo com o sucessor é aproximadamente igual ao intervalo com o anterior.

Já a figura \ref{fig:poicare_tn} apresenta um verdadeiro negativo. Embora os pontos estejam igualmente dispersos, o último batimento aparece mais deslocado para o canto inferior direito quando comparado aos dois casos anteriores. Essa diferença pode ter contribuído para que o modelo confundisse o caso negativo anterior como um batimento prematuro, resultando em um falso positivo.

Por fim, a figura \ref{fig:poicare_neg_maisfacil} mostra uma sequência normal do paciente 119. Diferentemente das anteriores, os batimentos formam agrupamentos mais concentrados, com o último batimento situado próximo ao centro e à linha de identidade.

Conforme visto na seção \ref{ch:resultados}, este \textit{fold} também foi o pior caso do modelo GRU e o mesmo ainda se saiu pior que o 
híbrido, com menor \textit{recall} e precisão. O que pode ser outra evidência da vantagem da CNN. Como a CNN atua como extrator de 
\textit{features} da morfologia, ela pode ter mitigado o impacto do ruído e da forma menos usual do ECG deste paciente, resultando 
em um desempenho melhor. Além disso, conforme visto, é possível que os intervalos RR tenha contribuído menos para a separação. Como ambos
os modelos receberam as mesmas \textit{features} RR, então a vantagem da extração feita pela CNN, pode ter ajudado o híbrido a se destacar.

\begin{figure}[H]
  \centering
  \caption{Scatter plot do paciente 203 de um falso positivo.}
   \includegraphics[width=0.6\textwidth]{figuras/analise_erros/poincare_paciente_mais_dificil_fp.png} 
  \label{fig:poicare_fp}
  \legend{Fonte: Elaborado pelo autor.}
\end{figure}

\begin{figure}[H]
  \centering
  \caption{Scatter plot do paciente 203 de um verdadeiro positivo.}
   \includegraphics[width=0.6\textwidth]{figuras/analise_erros/poincare_paciente_mais_dificil_tp.png} 
  \label{fig:poicare_tp}
  \legend{Fonte: Elaborado pelo autor.}
\end{figure}

\begin{figure}[H]
  \centering
  \caption{Scatter plot do paciente 203 de um verdadeiro negativo.}
   \includegraphics[width=0.6\textwidth]{figuras/analise_erros/poincare_paciente_mais_dificil_tn.png} 
  \label{fig:poicare_tn}
  \legend{Fonte: Elaborado pelo autor.}
\end{figure}

\begin{figure}[H]
  \centering
  \caption{Scatter plot do paciente 119 de uma sequência normal.}
   \includegraphics[width=0.7\textwidth]{figuras/analise_erros/poincare_paciente_mais_facil_neg.png} 
  \label{fig:poicare_neg_maisfacil}
  \legend{Fonte: Elaborado pelo autor.}
\end{figure}

Para visualizar o quão confiante o modelo foi em seus erros, na figura \ref{fig:curva_calibracao_pior_paciente} está a curva de calibração para esse paciente.
É possível observa um cenário de superconfiança, pois a curva está abaixo da diagonal. 
Por exemplo, considerando que em um grupo, o modelo preveja que em média, 90\% deles serão arrítmicos, a realidade é que menos de 60\% deles são. É possível observar,
também, como a superconfiança é mais intensa para \textit{bins} abaixo dos 80\%.

Conforme a discussão da seção \ref{sec:metricas}, a calibração não é diretamente relacionada ao desempenho de um modelo medida pelas métricas usuais.
Assim, mesmo se o modelo fosse muito sensível e tivesse uma precisão baixa, caso ele fosse bem calibrado, o tomador de decisão poderia 
confiar que as probabilidades previstas seriam mais realista.

Entretanto, existem técnicas que podem calibrar modelos. Em seu trabalho, \citeonline{mizilPredictingGoodProbs} apontou alguns exemplos como 
Calibração de Platt e regressão isotônica que são apropriados para classificação binária.

Já a explicabilidade, existem técnicas \textit{post-hoc} como LIME \cite{ribeiroWhyShouldTrustYou} ou SHAP \cite{lundbergSHAPE}.
Tais técnicas são aplicadas aos modelos já treinados e buscam identificar quais \textit{features} foram mais relevantes 
para a classificação; permitindo que o tomador de decisão entender o "raciocínio" do modelo.

\begin{figure}[H]
  \centering
  \caption{Curva de calibração para o paciente 203.}
   \includegraphics[width=0.8\textwidth]{figuras/analise_erros/curva_calibracao_pior_paciente.png} 
  \label{fig:curva_calibracao_pior_paciente}
  \legend{Fonte: Elaborado pelo autor.}
\end{figure}


