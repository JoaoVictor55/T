\chapter{Introdução}
\section{Contextualização}
\label{sec:contextualizacao}

Segundo a Organização Mundial da Saúde (\citeonline{who_cvds_2025}), as doenças cardiovasculares estão entre as principais causas de morte no mundo. Em 2022, foram responsáveis por aproximadamente 32\% de todos os óbitos globais. Ainda segundo \citeonline{agenciabrasil_cvds_2024}, as arritmias cardíacas causam cerca de 320 mil mortes súbitas anualmente.

Um dos principais exames utilizados no diagnóstico de doenças cardiovasculares é o eletrocardiograma (ECG), obtido por meio de eletrodos posicionados na superfície da pele para medir as correntes elétricas que controlam as contrações do músculo cardíaco. A partir desse exame, é possível identificar diversas alterações, incluindo arritmias, caracterizadas por perturbações na condução ou na origem dos impulsos elétricos cardíacos.

Nos últimos anos, métodos de aprendizagem de máquina vêm sendo empregados para auxiliar no diagnóstico de arritmias com o objetivo de otimizar o processo e apoiar o trabalho dos especialistas. Entretanto, a aplicação dessas técnicas não é trivial. O ritmo cardíaco apresenta grande variabilidade entre indivíduos — e até dentro de um mesmo indivíduo — tanto em condições normais quanto patológicas. Além disso, o sinal de ECG é naturalmente sujeito a ruídos e distorções.

Essa variabilidade torna a detecção automática de arritmias um desafio pois o modelo precisa aprender características que sejam generalizáveis a diferentes pacientes e condições. Soma-se a isso o desafio de projetar o próprio sistema de detecção: é necessário avaliar cuidadosamente decisões relacionadas a pré-processamento, seleção de \textit{features}, arquitetura do modelo e escolha de métricas, garantindo que estas sejam adequadas ao contexto clínico do problema.

Diante desse cenário, o presente trabalho tem como objetivo aplicar um modelo de aprendizado profundo para a detecção de arritmias em sinais de ECG, investigando as decisões metodológicas envolvidas e analisando como interpretar adequadamente os resultados obtidos.

Vale destacar que o foco deste trabalho não é propor um novo método ou alcançar desempenho de estado da arte, mas compreender o processo de modelagem, as decisões envolvidas e como interpretar os resultados obtidos.

\section{Problema de pesquisa}

\label{sec:problema_pesquisa}

O desafio da aplicação de modelos de aprendizagem de máquina em sinais de ECG envolve tanto a busca por um bom desempenho quanto a definição das decisões metodológicas adequadas. Para isso, é fundamental compreender o domínio do problema, identificar quais \textit{features} são relevantes, selecionar as métricas apropriadas e avaliar como essas escolhas influenciam o processo de modelagem.

Grande parte da complexidade surge da necessidade de conciliar dois domínios distintos: a cardiologia, que fornece o conhecimento clínico necessário para interpretar o sinal, e a aprendizagem de máquina, que disponibiliza as ferramentas computacionais para modelá-lo. Essa interseção exige entender os campos para formular o problema de modo coerente e avaliar corretamente os resultados.

\section{Objetivos}

Diante desses desafios, estabeleceram-se os seguintes objetivos.

\subsection{Objetivo geral}

Aplicar um modelo de aprendizagem profunda para a classificação de arritmias cardíacas em sinais de ECG e compreender os aspectos metodológicos envolvidos nesse processo.
\subsection{Objetivos específicos}

\begin{itemize}

    \item \textbf{Compreender o eletrocardiograma e as arritmias cardíacas}: estudar a morfologia do ECG e sua relação com o funcionamento cardíaco, identificando padrões associados a diferentes arritmias.
    \item \textbf{Analisar uma base de dados de ECG}: identificar uma base de dados e entender a  sua estrutura, incluindo suas anotações e a relação destas com as patologias estudadas.
    \item \textbf{Investigar técnicas de pré-processamento}: avaliar quais procedimentos são necessários para preparar o sinal antes de sua utilização em modelos baseados em redes neurais.
    \item \textbf{Estudar arquiteturas de redes neurais aplicadas ao problema}: analisar como diferentes tipos de redes podem explorar características do ECG.
    \item \textbf{Avaliar métricas adequadas ao desempenho do modelo}: compreender como cada métrica contribui para a interpretação do modelo no contexto da classificação de arritmias.
    \item \textbf{Realizar experimentos de classificação de arritmias em ventricular e normal}: aplicar uma arquitetura de rede neural ao problema e avaliar seu desempenho por meio de experimentação.
\end{itemize}

\section{Justificativa}

O uso de modelos de aprendizagem de máquina para auxiliar no diagnóstico de arritmias cardíacas apresenta grande potencial, tanto pela relevância clínica dessas condições quanto pela capacidade desses modelos de lidar com grandes volumes de dados e com a variabilidade inerente ao sinal de ECG. No entanto, aplicar tais métodos de forma eficaz exige uma compreensão adequada do domínio clínico e de como ele se relaciona com as técnicas de inteligência artificial.

No caso do ECG, a diversidade de arritmias, a variabilidade fisiológica entre pacientes e a presença de ruídos tornam o problema especialmente desafiador \cite{saadatnejad2020}. Esses fatores reforçam a necessidade de compreender não apenas o desempenho de um modelo, mas também o processo de construção, preparação e avaliação envolvidos.

Dessa forma, a justificativa deste trabalho reside na análise desse processo: entender quais decisões metodológicas influenciam o comportamento de modelos de aprendizagem profunda aplicados à classificação de arritmias. Vale destacar que o foco não está em propor um novo método ou atingir resultados de estado da arte, mas em discutir os desafios envolvidos e as possíveis soluções adotadas ao longo da modelagem.

Espera-se que as reflexões apresentadas contribuam para uma compreensão mais clara das dificuldades práticas do problema e auxiliem futuros estudos ou aplicações na área.
\section{Organização}

O presente trabalho está organizado da seguinte forma. No capítulo \ref{ch:fundamentacao} é explicado sobre 
arritmias cardíacas, começando sobre o funcionamento do coração e a caracterização das arritmias e, em seguida,
sobre os modelos de redes neurais usadas. Além de uma revisão de outros trabalhos sobre o tema.
No capítulo \ref{ch:metodologia} é explicado sobre a metodologia usada 
e o banco, o processo de limpeza e arquiteturas das redes. No capítulo \ref{ch:resultados}, é discutido os resultados 
alcançados pelos modelos através das métricas empregadas e quais problemas foram identificados. No 
capítulo \ref{ch:analise_erros_pior_fold} é feita uma breve investigação do pior caso do melhor modelo. Por fim, no capítulo
\ref{ch:conclusao} é feita a conclusão do trabalho com a discussão de trabalhos futuros.