\chapter{Conclusão}
O presente trabalho buscou compreender o processo de aplicação de redes neurais na classificação de arritmias ventriculares. Para isso, foram avaliados dois modelos — um baseado apenas em GRUs e outro híbrido, combinando CNN com GRUs — de modo a observar como diferentes arquiteturas lidam com a extração de padrões locais e dependências temporais do ECG.

A análise das métricas evidenciou que o modelo híbrido apresentou melhor desempenho, embora sujeito a certo grau de overfit. Para aprofundar a compreensão sobre esse resultado, foi realizada uma investigação do pior caso do modelo, revelando que a falha ocorreu em um paciente com sinais atípicos, porém clinicamente plausíveis. Além disso, identificou-se o problema de superconfiança, um desafio importante para aplicações em cenários críticos, embora não exclusivo deste trabalho.

De modo geral, o objetivo de explorar e compreender o processo de aplicação de modelos de deep learning para o ECG foi alcançado, permitindo identificar limitações práticas, desafios do domínio e potenciais caminhos de melhoria.

Durante a análise foi identificado outros problemas como superconfiança da rede que é um empecilho para uma adoção em 
cenário real, embora tais problemas não sejam restritos a esse modelo. 

\section{Trabalhos futuros}

Como trabalhos futuros, sugere-se aprofundar a análise do pior cenário, incluindo a avaliação dos demais sinais disponíveis e o uso de métodos de explicabilidade, como o LIME, para compreender de forma mais clara os padrões que levaram à falha do modelo.

Considerando que a abordagem híbrida obteve os melhores resultados, recomenda-se investigar novas arquiteturas, especialmente variantes bidirecionais das redes recorrentes. Outro ponto importante é o tratamento do desbalanceamento, que pode ser abordado por meio de geração de dados sintéticos ou estratégias baseadas em redes concorrentes.

Após a superação dessas limitações e a obtenção de resultados mais robustos, propõe-se realizar a avaliação final no conjunto Ds2 e, posteriormente, aplicar técnicas de calibração para produzir probabilidades confiáveis, bem como métodos adicionais de explicabilidade. Essas etapas visam não apenas melhorar o desempenho do modelo, mas também aumentar sua confiabilidade para aplicação em um cenário clínico.