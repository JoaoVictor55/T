\setlength{\absparsep}{18pt}
\begin{resumo}

O presente trabalho teve como objetivo principal aplicar modelos de aprendizagem profunda para a classificação de arritmias cardíacas ventriculares em sinais de ECG. O foco não foi alcançar o estado da arte, mas sim compreender o processo metodológico envolvido, investigando as técnicas de pré-processamento, as arquiteturas de redes neurais e as métricas de avaliação adequadas ao problema.

Para isso, foi utilizado o banco de dados MIT-BIH \textit{Arrhythmia Database}, focando na classe ventricular (V). O pré-processamento incluiu a limpeza de ruídos do sinal, segmentação e reamostragem dos batimentos para 288 amostras, além da inclusão dos intervalos RR como \textit{features}. Foram comparadas duas arquiteturas: um modelo GRU puro e um modelo híbrido (CNN + GRU). Ambos foram treinados com sequências de 16 batimentos e avaliados usando uma estratégia interpaciente com validação cruzada de 5 \textit{folds} no conjunto DS1.

Os resultados indicaram que o modelo híbrido (CNN + GRU) teve um desempenho superior ao do GRU puro, atingindo um \textit{f1-score} médio de 0,8593 e \textit{recall} de 0,8726, em comparação com 0,8060 de \textit{f1-score} do modelo GRU. Embora ambos os modelos tenham demonstrado sobreajuste, o fenômeno foi menos intenso no modelo híbrido. Uma análise de erros do pior cenário revelou que 98,46\% das falhas se concentraram em um único paciente (203), notório por sua complexidade. Além disso, identificou-se um problema de superconfiança nas previsões do modelo.

Em conclusão, o trabalho atingiu seu objetivo de explorar e compreender o processo de aplicação de \textit{deep learning} ao ECG. A superioridade do modelo híbrido sugere que a estratégia de usar CNNs para extração de características morfológicas combinada com GRUs para análise temporal é vantajosa. A análise dos erros, do \textit{overfitting} e da superconfiança evidenciou os desafios práticos e as limitações atuais para a aplicação desses modelos em cenários clínicos reais, principalmente devido à alta variabilidade dos sinais entre pacientes.

\vspace{\onelineskip}
\noindent 
\textbf{Palavras-chave}: aprendizado de máquina, eletrocardiograma, redes neurais.
\end{resumo}

% resumo em inglês
\begin{resumo}[Abstract]
 \begin{otherlanguage*}{english}
     The main objective of this study was to apply deep learning models to classify ventricular cardiac arrhythmias in ECG signals. The focus was not on achieving state-of-the-art performance, but rather on understanding the methodological process involved, investigating preprocessing techniques, neural network architectures, and evaluation metrics appropriate to the problem.

      For this purpose, the MIT-BIH Arrhythmia Database was used, focusing on the ventricular (V) class. Preprocessing included signal noise cleaning, beat segmentation and resampling to 288 samples, as well as the inclusion of RR intervals as features. Two architectures were compared: a pure GRU model and a hybrid model (CNN + GRU). Both were trained with sequences of 16 beats and evaluated using an interpatient strategy with 5-fold cross-validation on the DS1 dataset.

      The results indicated that the hybrid model (CNN + GRU) outperformed the pure GRU model, achieving an average f1-score of 0.8593 and a recall of 0.8726, compared to the GRU model's f1-score of 0.8060. Although both models demonstrated overfitting, the phenomenon was less pronounced in the hybrid model. A worst-case error analysis revealed that 98.46\% of the failures were concentrated in a single patient (203), notable for its complexity. Furthermore, an overconfidence problem in the model's predictions was identified.

      In conclusion, this work achieved its objective of exploring and understanding the process of applying deep learning to ECG. The superiority of the hybrid model suggests that the strategy of using CNNs for morphological feature extraction combined with GRUs for temporal analysis is advantageous. The analysis of errors, overfitting, and overconfidence highlighted the practical challenges and current limitations for applying these models in real-world clinical settings, mainly due to the high variability of signals among patients.
    \vspace{\onelineskip}
    
    \noindent 
    \textbf{Keywords}: machine learning, electrocardiography, neural networks.
 \end{otherlanguage*}
\end{resumo}
