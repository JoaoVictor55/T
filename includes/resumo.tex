\setlength{\absparsep}{18pt}
\begin{resumo}

O presente trabalho teve como objetivo principal aplicar modelos de aprendizagem profunda para a classificação de arritmias cardíacas ventriculares em sinais de ECG. O foco não foi alcançar o estado da arte, mas sim compreender o processo metodológico envolvido, investigando as técnicas de pré-processamento, as arquiteturas de redes neurais e as métricas de avaliação adequadas ao problema.

Para isso, foi utilizado o banco de dados MIT-BIH Arrhythmia Database, focando na classe ventricular (V). O pré-processamento incluiu a limpeza de ruídos do sinal, segmentação e reamostragem dos batimentos para 288 amostras, além da inclusão dos intervalos RR como features. Foram comparadas duas arquiteturas: um modelo GRU puro e um modelo híbrido (CNN + GRU). Ambos foram treinados com sequências de 16 batimentos e avaliados usando uma estratégia interpaciente com validação cruzada de 5 folds no conjunto DS1.

Os resultados indicaram que o modelo híbrido (CNN + GRU) teve um desempenho superior ao do GRU puro, atingindo um F1-Score médio de 0,8593 e recall de 0,8726, em comparação com 0,8060 de F1-Score do modelo GRU. Embora ambos os modelos tenham demonstrado sobreajuste, o fenômeno foi menos intenso no modelo híbrido. Uma análise de erros do pior cenário revelou que 98,46\% das falhas se concentraram em um único paciente (203), notório por sua complexidade. Além disso, identificou-se um problema de superconfiança nas previsões do modelo.

Em conclusão, o trabalho atingiu seu objetivo de explorar e compreender o processo de aplicação de deep learning ao ECG. A superioridade do modelo híbrido sugere que a estratégia de usar CNNs para extração de características morfológicas combinada com GRUs para análise temporal é vantajosa. A análise dos erros, do \textit{overfitting} e da superconfiança evidenciou os desafios práticos e as limitações atuais para a aplicação desses modelos em cenários clínicos reais, principalmente devido à alta variabilidade dos sinais entre pacientes.

\vspace{\onelineskip}
\noindent 
\textbf{Palavras-chave}: aprendizado de máquina, eletrocardiograma, redes neurais.
\end{resumo}

% resumo em inglês
\begin{resumo}[Abstract]
 \begin{otherlanguage*}{english}
     
    \vspace{\onelineskip}
    
    \noindent 
    \textbf{Keywords}: 
 \end{otherlanguage*}
\end{resumo}
