\setlength{\absparsep}{18pt}
\begin{resumo}

O trabalho focou na aplicação de modelos de deep learning para classificar arritmias cardíacas ventriculares (classe V) em sinais de ECG do banco de dados MIT-BIH, com o objetivo de entender a metodologia, e não apenas alcançar o estado da arte. Após pré-processamento (incluindo limpeza, segmentação, reamostragem para 288 amostras e uso de intervalos RR como features), duas arquiteturas foram comparadas: um modelo GRU e um modelo híbrido (CNN + GRU), treinados com sequências de 16 batimentos e avaliados com validação cruzada interpaciente. O modelo híbrido demonstrou desempenho superior, atingindo um f1-score médio de 0,8593 (contra 0,8060 do GRU puro), sugerindo a vantagem de combinar CNNs para extração de características morfológicas com GRUs para análise temporal. Apesar da superioridade, ambos os modelos apresentaram sobreajuste e superconfiança, sendo a maioria dos erros concentrada em um paciente específico (203), o que destacou os desafios práticos da alta variabilidade interpaciente para a aplicação clínica real desses modelos.

\vspace{\onelineskip}
\noindent 
\textbf{Palavras-chave}: aprendizado de máquina, eletrocardiograma, redes neurais.
\end{resumo}

% resumo em inglês
\begin{resumo}[Abstract]
 \begin{otherlanguage*}{english}
  
    This work focused on applying deep learning models to classify ventricular cardiac arrhythmias (class V) in ECG signals from the MIT-BIH database, aiming to understand the methodology, not just achieve state-of-the-art results. After preprocessing (including cleaning, segmentation, resampling to 288 samples, and using RR intervals as features), two architectures were compared: a GRU model and a hybrid model (CNN + GRU), trained with 16-beat sequences and evaluated with inter-patient cross-validation. The hybrid model demonstrated superior performance, achieving an average f1-score of 0.8593 (versus 0.8060 for the pure GRU), suggesting the advantage of combining CNNs for morphological feature extraction with GRUs for temporal analysis. Despite their superiority, both models showed overfitting and overconfidence, with most errors concentrated in a specific patient (203), which highlighted the practical challenges of high interpatient variability for the real clinical application of these models.

    \vspace{\onelineskip}
    
    \noindent 
    \textbf{Keywords}: machine learning, electrocardiography, neural networks.
 \end{otherlanguage*}
\end{resumo}
