\begingroup % Cria um grupo local para as configurações de fonte
\centering
\footnotesize 

% Definição de larguras
% Calculando as larguras para a coluna p{} manual
% Largura restante = \textwidth - (Largura de l) - (Bordas)
% Colunas: l + p1 + p2 + p3 + p4 + p5
% Vamos alocar 10% para o Trabalho, e 18% para o restante (5 * 0.18 = 90%).

% Coluna Trabalho (Ano) - 10% da largura do texto
\setlength{\colwidthA}{0.18\textwidth} 

% As 5 colunas restantes dividem o 90% restante. 
% (0.9 * \textwidth) / 5 = 0.18\textwidth (aproximadamente)
\setlength{\colwidthB}{0.18\textwidth} % Tipo de Modelo
\setlength{\colwidthC}{0.18\textwidth} % Particionamento
\setlength{\colwidthD}{0.18\textwidth} % Features Incluídas
\setlength{\colwidthE}{0.18\textwidth} % Segmentação do Batimento e Limpeza

% Usamos l (para a primeira) e p{largura} com \raggedright para as demais.
\begin{longtable}{|>{\RaggedRight}p{\colwidthA}|>{\RaggedRight}p{\colwidthB}|>{\RaggedRight}p{\colwidthC}|>{\RaggedRight}p{\colwidthD}|>{\RaggedRight}p{\colwidthE}|>{\RaggedRight}p{\colwidthE}|}
% A RaggedRight (do pacote ragged2e, que você tem) é a versão de parágrafo de \raggedright.

% ---------------------------------
% Definição do Cabeçalho da Tabela
% ---------------------------------

\caption{Resumo dos desempenho dos modelos para a classificação de arritmia (ventricular)}
\label{tab:resultados_trabalhos}\\
\hline
\textbf{Trabalho (Ano)} & \textbf{Dataset} & \textbf{Sensibilidade (\%)} & \textbf{Precisão (\%)} \\
\hline
\endhead % Fim do Cabeçalho (Repete em todas as páginas)

% --------

\multicolumn{6}{|r|}{Continua na próxima página...} \\
\hline
\endfoot % Fim do Rodapé (Aparece em todas as páginas, exceto na última)

% ---------------------------------
% Definição do Rodapé da Última Página
% ---------------------------------
\hline
\endlastfoot % Fim da Tabela

% ---------------------------------
% Conteúdo da Tabela
% ---------------------------------

% O \makecell foi removido, exceto nas células onde há uma quebra manual forte
% mas o longtable com p{} já quebra automaticamente.
% Vamos manter o \makecell[l] para a primeira coluna para garantir o alinhamento
% e a quebra manual (se necessário).
\makecell[l]{de Chazal et al. \\ (2004)} & DS2. & 77,7. & 81,9. \\
\hline
\makecell[l] {Mousavi et al. \\(2019)} & DS2. & 98,98. & 97,40.\\
\hline
\makecell[l] {Li et al.\\(2019)} & DS2. & 92,5.& 97,40.\\
\hline
\makecell[l] {Saadatnejad et al.\\(2020)} & Intrapaciente. & 98,22.& 92,97.\\
\hline
\makecell[l] {Kiranyaz et al. \\(2016)} & Intrapaciente. & 95,0.& 89,5.\\
\hline
\makecell[l] {TCC (GRU)} & Cross-validação 5 folds (ds1). & 80,39.& 85,15.\\
\hline
\makecell[l] {TCC (Híbrido)} & Cross-validação 5 folds (ds1). & 87,26.& 88,0.\\
\end{longtable}
\endgroup